\documentclass{scrartcl} % KOMA-Script class for automatic colors
\usepackage[sexy]{evan} % Load evan.sty with 'sexy' option
\usepackage{amsmath}    % For math formatting
\usepackage{xcolor}     % Ensure xcolor is available (often needed)
\newcommand{\Tau}{\mathcal{T}}
\newcommand{\sP}{\mathcal{P}}
\newcommand{\sF}{\mathcal{F}}
\newcommand{\sI}{\mathcal{I}}
\newcommand{\sD}{\mathcal{D}}
\newcommand{\sS}{\mathcal{S}}
\newcommand{\sC}{\mathcal{C}}
\newcommand{\sB}{\mathcal{B}}
\newcommand{\sSp}{Sierpi\'nsky}
\newcommand{\ssp}{sierpi\'nsky}
\title{Point Set Topology}
\author{Swarnava Chattaraj}
\date{\today}

\begin{document}

\maketitle
\section{Axioms of Topology and Examples}
Let $X$ be any set, $\sP(x)$ denote the power
set of $X$. A family $\Tau \subset \sP(x)$ is called a
\textbf{topology} on $X$ if the following properties are
satisfied :
\begin{itemize}
	\item $\emptyset$, $X$ are in $\Tau$.(Not really
		required)
	\item $\left(\textbf{AU}\right)$
		If $\sF$ is in $\Tau$, then
		$\bigcup \{U : U \in \sF \} \in \Tau$
	\item $\left(\textbf{FI}\right)$ If $\sF$ is a
		finite subfamily of $\Tau$, then 
	 	$\bigcap \{U : U \in \sF \} \in \Tau$
\end{itemize}
Due to pedagogical reasons and sheer neglect we will
replace $\left(X, \Tau \right)$ is a\\ 
\textbf{topological space} with just $X$ is a topological space and the topology induced is understood form the 
context.
\begin{definition}[Open and Closed Sets]
	Elements of $\Tau$ are called \textbf{open} 
	sets in $\left(X, \Tau\right)$. A subset 
	$\sF \subset X$ is called \textbf{closed} if
	its complement $\sF^c \coloneq X\setminus\sF$
	is open in $\left(X, \Tau\right)$.
\end{definition}
\begin{remark}
We say that we have determined a Topological space 
completely if we know all of its open sets or 
equivalently all of its closed sets.
\end{remark}
\begin{lemma}[DeMorgan's Law]
Recall from basic set theory that $\left(A \cup B\right)
^c = A^c \cap B^c$
\end{lemma}
I am hoping it is clear why we are bringing this result
, as it will often be a useful trick to translate our 
argument from the context \textbf{open sets} to
\textbf{closed sets}.

\begin{definition}[Neighbourhoods]
Let $X$ be a topological space. A subset $A$ of $X$ is
called a \textbf{neighbourhood} of $x \in X$ if there
exists a open set $U$ such that $x \in U \subset A$.\\
$A$ will be termed an \textbf{open neighbourhood} if $A$
itself is \textbf{open}.
\begin{remark}
A subset $A$ is open iff it is a neighbourhood of every point belonging to it.
\end{remark}
\begin{proof}
The first direction, \\
$x \in A, A \subset A \implies A$ is a neighbourhood of all points in $A$.\\
The other direction, \\
if $\forall\ x \in A$, is a neighbourhood of $x$ that
implies that $\exists\ U_x \forall\ x \in A : U_x 
\subset A$. Taking the union we have $\bigcup_{x \in A}
U_x = A$.
\end{proof}
\end{definition}
\newpage
\begin{remark}
Well we made a note that the first axiom of the
Topological space was not required, we will only hint to
it. With enough set theory background the following two
statements can be made. Let $U \subset A$,
\begin{itemize}
	\item $\bigcap\limits_{U \in \emptyset} U = X$
		(\textbf{FI})
	\item $\bigcup\limits_{U \in \emptyset} U =
		\emptyset$ (\textbf{AU})
\end{itemize}
Well that actually completes the claim that the first
axiom was not really necessary.\\ Although we will not
spend anymore time on this, the two set theoretic
statement are to be contemplated on and will be used 
later on.
\end{remark}

\begin{theorem}
Let $\left(X, d\right)$ be a metric space, $\Tau(d)$
defined as, 
\[
	\Tau(d) \coloneq \{ U \subset X : U = 
	\bigcup\limits_i B_{r_i} (x_i) \}
\]
defines a topology on $X$. Generally $\left(X, \Tau(d)
	\right)$ will be called the topological space
associated with metric $d$ and $\Tau(d)$ as the topology
associated with metric $d$. Such topology that arise
from a metric are called \textbf{Metric Topology} or 
\textbf{Metrizable Topology}.
\end{theorem}
\begin{remark}
Consider $\left(\RR^n, \Tau(d_2)\right)$. This is referred as the \textbf{n-dimensional euclidean space with its usual
topology}. It has more interesting properties which we will discover as we go along and by $d_2$ we actually mean the
\textbf{euclidean metric}.
\end{remark}
\begin{lemma}
	Consider $\left(\RR^n, \Tau(d_2)\right)$. Let $\left(a,b\right) \coloneq \{x \in \RR : a < x < b \}$. 
	Then, 
\[
	\mathcal{K} \coloneq \prod\limits_{i=1}^n \left(a_i, b_i\right) \in \Tau(d_2).
\]
\end{lemma}
\begin{proof}
	The idea is to show that for every point $x \in \mathcal{K}$, there is some open ball around $x$
	the that is contained in $\mathcal{K}$. 
Let $x = \left(x_1,\cdots,x_n\right) \in \mathcal{K}$ then $a_i < x_i < b_i$.
Here we define,
\[
	\delta_i \coloneq \min_i \{|x_i - a_i|, |x_i - b_i|\}
\]
And then we choose our radius $r'$ to be,
\[
	r'_x \coloneq \dfrac{\min\limits_i\delta_i}{\sqrt{n}}
\]
Note that the choice of re-scaling constant $\sqrt{n}$ is specific to the choice of metric, which in our case is $d_2$. Now take a ball $B_{r'}$, it is easy to verify that $B_{r'_x} \subset \mathcal{K}$.
We conclude our argument as follows,
\[
   \mathcal{K} = \bigcup\limits_{x \in \mathcal{K}} B_{r'_x}(x) \in \Tau(d_2)
\]
The lemma is true for any other metric $d$, except one would have to take care of the re-scaling factor in the proof.
\end{proof}
\begin{example}[Discrete and Indiscrete Topology]
	\label{ex:di}
Let $X$ be any set. Let $\Tau = \sI \coloneq \{\emptyset
, X\}$ and $\Tau = \sD \coloneq \sP(x)$, then 
$\left(X, \sI\right)$ and $\left(X, \sD\right)$ form
indiscrete and discrete topological space.
\end{example}
\begin{example}[Sierpi\'nsky Space]
\label{ex:sierpspace}
	Let $X = \{0, 1\}$, and
	$\Tau = \{\emptyset, \{0\}, \{0, 1\} \}$.\\
Then $\sS \coloneq \left(X, \Tau \right)$
	is known as the \textbf{Sierpi\'nsky Space}. \\
	After we have defined what is a continuous 
	functions we will prove that we can completely
	determine any topological space $Y$ by the
	set of all continuous functions from
	$Y$ to $\sS$.
\end{example}
\begin{example}[Generalizsed Sierpi\'nsky Point]
	\label{ex:gsierspace}
	Take any topological space 
	$\left(X, \Tau\right)$, now consider 
	$s(X) \coloneq
	\left( X \sqcup \{*\}, \Tau'\right)$ where \\
	$\Tau' = \Tau \cup \{X \cup \{*\} \}$. Then
	thing to notice here is the only open set
	containing\\ $*$ is $X \cup \{*\}$. A space
	that satisfies the mentioned property is called
	a\\ \textbf{Generalised \sSp \ space}. A point 
	$x \in X$ is called a \textbf{\ssp \ point}
	if the only open set containing it is $X$.
\end{example}
\begin{remark}
	Recall \autoref{ex:sierpspace} according 
	the definition
	of \ssp \ point\\ in \autoref{ex:gsierspace}, $\{1\}$
	is a \ssp \ point because the only open set
	containing it is $X$.
\end{remark}
\newpage

\section{Digression to the land of Algebra}
Let $X$ be any set and $\FF$ is $\RR$ or $\CC$. 
Let $\sF \coloneq F\left(X,\FF\right)$ be the set of
all functions $f : X \rightarrow \FF$. Now the
 set already has the following structures to $\sF$
already,
\begin{itemize}
	\item $\left(f + g\right)(x) = f(x) + g(x)$
	\item $\left(\alpha f\right)(x) = \alpha f(x)$
\end{itemize}
for all $x \in X$, $f, g \in \sF$, $\alpha \in \FF$.\\
So $\sF$ is a \textbf{Vector Space} over the field
$\FF$.
It has one more interesting algebraic structure,
\[
\left(f\circ g \right)(x) \coloneq f(x)g(x) \in \sF 
\]
The multiplication is known as the \textbf{point-wise
multiplication}. It has the following
properties \\
\textbf{(i) associative},
$\left(\left(f\circ g \right)\circ \left(h\right)\right) (x) = \left(\left(f\right)\circ\left(g\circ h
	\right)\right)(x)$\\
\textbf{(ii) commutative},
$\left(f\circ g\right)(x) = \left(g\circ f \right)(x)$.
\\
\textbf{(iii) bilinear with respect to the
vector space structure} of $\sF$,
\[
f\circ \left(g + h\right) = f\circ g + f\circ h = g\circ f + 
h\circ f =
\left(g + h\right)\circ f
\]
\[
	f\circ\left(\alpha g\right)  = 
	\alpha \left(f\circ g\right) 
\]
\[
\left(\alpha f\right)\circ g = \alpha \left( f\circ g\right) 
\]
The bilinearity is tedious to check.\\
Now the $+$ operation of the vector space structure of 
$\sF$ is \textbf{associative}, \textbf{commutative} and
supports \textbf{additive inverse}.
The \textbf{0(x)} function is already a continuous 
function so $0(x) \in \sF$.
So $\left(\sF, +\right)$ is an \textbf{Abelian group}.\\
Now $1(x)$ or the identity function is continuous
so it is in $\sF$. As indicated above
$\left(\sF, \circ \right)$, is associative, and
distirbutive.\\
So $\left(\sF, +, \circ \right)$ is a \textbf{Ring}
or more specifically a \textbf{unitary Ring}. Not only
that it is \textbf{Commutative Ring}. The dual structure
 of a vector space and ring along with the bilinearity 
 makes it also a 
\textbf{Commutative Algebra} over $\FF$.\\
Moreover if you restrict $\sF$ to 
$\sB \coloneq B(X,\FF)$ in which you
restrict yourself to only bounded functions and impose the \textbf{sup norm},
\[
	\|f\|_\infty \coloneq \sup\{|f(x)| : x \in X\}
\]
Due to the point-wise multiplication definition the 
following is also true,
\[
	\|fg\|_\infty \leq \|f\|_\infty\|g\|_\infty
\]
An algebra together with the above property is known as
the \textbf{Normed Algebra}.\\
Associated to the above norm is the following metric:
\[
d_\infty(f,g) = \|f - g\|_\infty
\]
Convergence of function with respect to the above metric
gives uniform convergence. We would come back to this
later on.\\
Now we restrict $X$ to be a metric space.Another 
subset of $\sB$, call it $\sC \coloneq C(X,\FF)$ be
the set of all continuous function, so $\sC$ contains
all 
continuous and bounded functions from $X$ to $\FF$.
Then it is easy to see that $\sC$ is a vector subspace
of $\sB$.\\
The property which we would study later on is that
$\sC, \sB$ are complete, and so they form a very 
important class of normed linear algebras knows as
\textbf{Banach Algebras}.
\newpage
\section{Mappings}
\begin{definition}[Continuity]
Let $\left(X_i, \Tau_i\right)$ for $i = 1,2$ be 
two topological spaces. We call a set theoretic function
$f : \left(X_1, \Tau_1\right) \rightarrow
\left(X_2, \Tau_2\right)$ is continuous function if
$f^{-1}(V) \in \Tau_1$ for all $V \in \Tau_2$.
\end{definition}
Not that anyone asked but I want to emphasize that 
$f^{-1}(x) \coloneq \{x \in X_1 : f(x) \in V \}$, if 
you want to be to pedantic you can replace $f^{-1}$ with
$f^{pre}$.
\begin{theorem}
Let $\left(X_i, d_i\right)$ for $i = 1,2$ be 
two metric spaces.
$f : \left(X_1, \Tau(d_1)\right) \rightarrow
\left(X_2, \Tau(d_2)\right)$ is continuous function iff
$f : \left(X_1, d_1\right) \rightarrow
\left(X_2, d_2\right)$ is continuous.
\end{theorem}
\begin{proof}
Just to recall the way $\Tau(d) \coloneq \{ 
U \subset X : U = \bigcup_i B_{r_i} (x_i) \}$.\\
The first direction,\\
It is evident from construction that for any 
$f(x) \in X_2$ take a open ball of radius $\epsilon > 0$
, $V = B_{\epsilon}(f(x)) \in \Tau(d_2)$.
By definition of continuity $f^{-1}(V) \in 
\Tau(d_1)$. Since $x \in f^{-1}(V)$, there exists at least
one open ball of radius $\delta > 0$, $U = B_{\delta}(x)\subset
f^{-1}(V)$ which contains $x$. Now pick a $y \in U$, then $f(y) \in V$ which is 
 equivalent to saying that 
 if $d_1(y,x) < \epsilon \implies d_2(f(y),f(x)) < \epsilon$.\\
The opposite direction,\\
Now to begin the proof, take any $V \in \Tau(d_2)$
take a point $x \in
X_1$ such that $f(x) \in V$. So we have $x \in f^{-1}(V)
$. Now $V$ is a open set so there is some open ball
of radius $\epsilon_x > 0$, such that
\[
	B_{\epsilon_x}(f(x))\subset V 
\]
The next step is to notice that for a given $x$,
$\ \forall \ \epsilon > 0, \ \exists \ \delta > 0$
such that\\
$f$ maps  $B_{\delta}(x)$ to $B_{\epsilon}(f(x))$ 
.This comes
from that fact that $f$ is continuous in the metric space
so 
\[d_1(y,x) < \delta \implies d_2(f(x),f(y)) < \epsilon
\]
So an appropriate $\delta_x > 0$
(more appropriately $\delta(x, \epsilon)$) can be chosen such that
\[
f(B_{\delta_x}(x)) \subset V \implies
B_{\delta_x}(x) \subset f^{-1}(V)
\]
And finally we will use the fact that we had chosen a
generic $x$ and a generic $V$ so,
\[
	\forall\ x \in f^{-1}(V),\ \  
	\exists \ \ B_{\delta_x}(x) \subset f^{-1}(V) 
\]
which means that 
\[
	f^{-1}(V) = \bigcup\limits_{x \in f^{-1}(V)}
	B_{\delta_{x}}(x)
\]
So $f^{-1}(V) \in \Tau(d_1)$.
\end{proof}
\newpage
\begin{theorem}\label{thm:composition}
Let $f : \left(X_1, \Tau_1\right) \rightarrow
\left(X_2, \Tau_2\right)$ and
$g : \left(X_2, \Tau_2\right) \rightarrow
\left(X_3, \Tau_3\right)$ be 
continuous functions. 
Then $g \circ f : \left(X_1, \Tau_1\right)
\rightarrow \left(X_3, \Tau_3\right)$ is also continuous.
\end{theorem}
\begin{proof}
The following set-theoretic results immediately kills the
proof, let $U$ be a set,
\[
	f^{-1}\left(g^{-1}(U)\right) = 
	\left(g \circ f\right)^{-1}(U)
\]
Let $V \in \Tau_3$, then we want to show that 
$\left(g \circ f\right)^{-1}(V) \in \Tau_1$. Since 
$g$ is continuous $g^{-1}(V) \in \Tau_2$ and since $f$ is continuous $f^{-1}\left(g^{-1}(V)\right) \in \Tau_1$ and 
we are done.
\end{proof}
\begin{definition}
We call $f : \left(X_1, \Tau_1\right) \rightarrow
\left(X_2, \Tau_2\right)$ a \textbf{homeomorphism} iff
$f$ is continuous and bijective and $f^{-1}$ is also
continuous. We call two topological spaces 
\textbf{homeomorphic} if there exists a homeomorphism
between the two spaces.
\end{definition}
\begin{remark}
$f : \left(X_1, \Tau_1\right) \rightarrow
\left(X_2, \Tau_2\right)$ be a homemorphism then we say $X_1 \overset{f}{\cong} X_2$ as a short hand for homeomorphic.\\
Also note that $X_2 \overset{f^{-1}}{\cong} X_1$. 
\end{remark}
\begin{remark}
Being homeomorphic is a equivalence relation in the 
collection of all topological spaces. So two homeomorphic topological space which are equivalent are in same 
equivalence class or more loosely are of \textbf{same type} or \textbf{same homeomorphism typle}.
\end{remark}
\begin{proof}
	The following properties of equivalence relation 
	need to be checked:
	\begin{itemize}
		\item symmetry is trivial because $f$ and
			$f^{-1}$ are both continuous.
		\item associativity is guaranteed by 
			\autoref{thm:composition}.
		\item reflexivity is guaranteed by the
			presence of identity map.(check that it is indeed a homeomorphism)
	\end{itemize}
\end{proof}

\begin{example}
The following sets are homeomorphic to each other:
\begin{itemize}
	\item Any two closed intervals containing more
		than one point.
	\item Any two non empty open intervals.
	\item Any non empty open interval and $\RR$.
\end{itemize}
For the first two one can always find homeomorphic maps
$f(x) = ax + b$, and for the third case it is a bit
tricky. One way to find such maps are composition. For
example homeomorphisms like $f(x) = 
\tan\left(\dfrac{\pi x}{2}\right)$ makes the intervals
$(-1,1)$ homeomorphic to $\RR$ and then use linear 
maps to make any open interval $(a,b)$ homeomorphic to 
$(-1,1)$. Then the composition is a homeorphism between
$(a,b)$ to $\RR$.
\end{example}
\begin{example}
	Given any topological space $X$, $\mathcal{H}(X)$
	is the set of all self-homeomorphisms.\\
	$\mathcal{H}(X)$ forms a \textbf{group}. In 
	particular $\mathcal{H}(\RR)$ is a huge group.
\end{example}
\begin{lemma}
A onto function $f : \RR \rightarrow \RR$ which is 
continuous and strictly monotone is a homeomorphism.
\end{lemma}
\begin{remark}
	The above theorem helps us characterize all the
	homeomorphisms of $\mathcal{H}(X)$.
\end{remark}
\begin{definition}
	We say that we can classify all topological spaces if we can create a set $\mathfrak{Top}$ of topological
	spaces such that
	\begin{itemize}
		\item if $X_1, X_2 \in \mathfrak{Top} \implies X_1 \ncong X_2$ 
		\item if $Y \notin \mathfrak{Top}$ then $\ \exists \ X \in \mathfrak{Top}$ such that
			$Y \cong X$.
	\end{itemize}
\end{definition}
\begin{remark}
	The construction of above $\mathfrak{Top}$ is hopeless. We might provide an answer very later because it 
	requires  decent group theory and even algebraic topology.
\end{remark}
\begin{definition}\label{eqn:topological invariant}
	A property $P$ is called a \textbf{topological property} or \textbf{topological invariant} if every 
	homeomorphism preserves it, i.e if $f : \left(X_1, \Tau_1\right) \rightarrow \left(X_2, \Tau_2\right)$
	be a homeomorphism then $\left(X_1, \Tau_1\right)$ satisfies $P$ iff $\left(X_2, \Tau_2\right)$. 
\end{definition}
\begin{remark}
To show that there is no homeomorphism between two topological spaces, find a property $P$ which is satisfied
by only one of them.
\end{remark}
\begin{definition}
We call $f: \left(X_1, \Tau_1\right) \rightarrow \left(X_2, \Tau_2\right)$ an open mapping if
\[
U \in \Tau_1 \implies f(U) \in \Tau_2
\]
We can similarly say $f$ is a closed mapping if
\[
	X_1\setminus K \in \Tau_1 \implies X_2 \setminus f\left(K\right) \in \Tau_2
\]
\newpage
\begin{lemma}\label{thm:bi+oimpliescont}
	If $f : \left(X_1, \Tau_1\right) \rightarrow \left(X_2, \Tau_2\right)$ is open and bijective then
	$f^{-1}: \left(X_2, \Tau_2\right) \rightarrow \left(X_1, \Tau_1\right)$ is continuous.
\end{lemma}
\begin{proof}
	For $f^{-1}$ to be continuous we want that $U \in \Tau_1 \implies \left(f^{-1}\right)^{-1}(U) \in \Tau_2$.\\
	Since $f$ is bijective $ \left(f^{-1}\right)^{-1}(U) \in \Tau_2 =  f(U) \in \Tau_2$. Notice that what we want
	is true because $f$ is open.
\end{proof}
\begin{remark}
	Extending \autoref{thm:bi+oimpliescont}, if $f$ itself is also continuous then $f$ is a \\homeomorphism.
\end{remark}
\end{definition}
\subsection{Metric Specific Notions of Equivalence}
\begin{definition}
We say two metric spaces $\left(X_i, d_i\right),\ i = 1,2$ are \textbf{topologically} \\ \textbf{equivalent}, if
	the underlying topological spaces $\left(X_i, \Tau(d_i)\right),\ i = 1,2$ are of the
	same homeomorphism type.
\end{definition}
\begin{definition}\label{def:similar}
	We say two metric space $\left(X_i, d_i\right),\ i = 1,2$ are \textbf{similar} if there is a 
	bijection $f: (X_1,d_1) \rightarrow (X_2,x_2)$ and it satisfies the following inequality, 
	\[
		c_1d_1(x,y) \leq d_2(f(x),f(y)) \leq c_2d_1(x,y) \ \forall\ x,y \in X_1
	\]
where $c_1, c_2$ be two positive real numbers.
\end{definition}
\begin{definition}
	Extending \autoref{def:similar}, if $c_1 = c_2 = 1$ then $\left(X_i, d_i\right), \forall\ i = 1,2$ are
	\textbf{isometric}. In our notes we will refer to such a $f:(X_1, d_1) \rightarrow (X_2, d_2)$ as
	an \textbf{isometry}.
\end{definition}
\begin{claim}\label{clm:3.20}
	$f$ in \autoref{def:similar} is a homeomorphsim between the underlying topology $\left(X_i, \Tau(d_i)\right),
\forall\ i = 1,2$.  
\end{claim}
\begin{proof}
So we want to show that $f$ is continuous and for that
we need that given any $x \in X_1, \ \forall\
\epsilon > 0, \ \exists\ \delta > 0 \  s.t$,
\[
	d_1(x,y) < \delta \implies
	d_2(f(x),f(y)) < \epsilon
\]
Which is trivial, for any given $\epsilon > 0$
just take 
$\delta = \dfrac{\epsilon}{c_2}$ and $ d_1(x,y) < \delta
$ then,
\[
d_2(f(x), f(y)) \leq c_2d_1(x,y) < c_2\delta = \epsilon
\]
Since $f$ is bijective it makes sense to talk about 
inverse at every point, so a similar argument can be 
made. Hence $f$ is a homeomorphism.
\end{proof}
\begin{remark}
	\autoref{clm:3.20} show that similarity is an 
	equivalence relation. Also
	\[
		\textbf{isometry} \implies 
		\textbf{similarity} \implies
		\textbf{topological equivalence}.
        \]
\end{remark}
\begin{remark}
	Let two similar metric spaces be $\left(X_1, d_1\right)$ and $\left(X_2, d_2\right)$ then a property
	P is called similarity invariant if
	\[
		\text{P holds for } \left(X_1, d_1\right) \iff \text{ P hold for }\left(X_2, d_2\right).
        \]
\end{remark}
\begin{definition}[Diameter and Bounded Sets]
	Let $\left(X,d\right)$ be a metric space, we then define \textbf{diameter} of the set $X$, by $\delta(X,d)$ as
	follows,
		\[
		\delta(X,d) \coloneqq \sup\{d(x,y) : x,y \in X\}
		\]
	We call the set $X$ bounded if $\delta(X) < \infty$. In that case we also call the metric $d$ a \textbf{bounded metric}. To emphasize there is
	no restriction that the $\delta(X)$ cannot be $-\infty$, we actually define 
	$\delta(\emptyset) \coloneqq -\infty$.
\end{definition}
\begin{example}
Let $f :\left(X_1, d_1\right) \rightarrow \left(X_2, d_2\right)$ be an isometry then, 
\[
\{d_1(x,y) : x,y \in X_1\} = \{d_2(a,b) : a,b \in X_2\}
\]
induced by $\left(f,f\right):X_1\times X_1 \rightarrow X_2\times X_2$.\\
\begin{proof}
	$f$ is a bijection between $X_1$ and $X_2$ that implies $(f,f)$ is a bijection between $X_1\times X_1$ and
	$X_2\times X_2$. Also $f$ is a isometry so $d_1(x,y) = d_2(f(x),f(y))$. Hence the two sets of distances are
	identical when we consider the map $(f,f)$.
\end{proof}
From our above reasoning it follows that 
\[
\delta(X_1,d_1) = \delta(X_2,d_2)
\]
So diameter is a isometric-invariant property.
\end{example}
\begin{example}
Let $d$ be a metric on some underlying set $X$. Now define a new metric $d'$ as follows,
\[
	d(x,y) = 2d'(x,y)
\]
Then consider the identity map $Id: (X,d) \rightarrow (X,d')$, then it is trivial to see that $(X,d)$ and $(X,d')$ are 
similar. But now if we look at the diameter then 
\[
	\delta(X,d) = 2\delta(X,d')
\]
So we conclude that diameter is not a similar-invariant property.
\end{example}
\end{document}

