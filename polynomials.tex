\documentclass{scrartcl} % KOMA-Script class for automatic colors
\usepackage[sexy]{evan} % Load evan.sty with 'sexy' option
\usepackage{amsmath}    % For math formatting
\usepackage{xcolor}     % Ensure xcolor is available (often needed)
% ===== Add this to your preamble =====
\usepackage{amsmath, amsthm, xcolor, thmtools}

\title{Sample Document with \texttt{evan.sty} Colors}
\author{Your Name}
\date{\today}

\begin{document}

\maketitle
\section{Introduction}

\begin{definition}[Binomial Polynomial]
    Let $k \in \NN$, we define the binomial polynomial as follows \\
    $\dbinom{x}{k} \coloneq \dfrac{1}{k!}x\left(x-1\right)\left(x-2 \right)\cdots\left(x - \left(k - 1\right)\right)$ for $k > 0$\\
    $\dbinom{x}{0} \coloneq 1$ for $k = 0$
\end{definition}


\begin{example}[CMI A9, 2020]
	A polynomial $P(x)$ of degree 7 satsifies $P(n) = 2^n$ for $n = 0$ to $7$. Find $P(10)$.
\end{example}
\begin{proof}[Solution]
	Consider the polynomial $G(x) \coloneq \dbinom{x}{0} + \dbinom{x}{1} + \cdots + \dbinom{x}{7}$.\\
	Now $P(x) = G(x)$ for $x = 0$ to $7$. Also degree of $G(x)$ is 7 and $P(x), G(x)$ agree on 8 points. We can happily conclude that $P(x) = G(x), \ \forall x$. So just calculate $G(10)$ which is trivial.
\end{proof}



\end{document}
