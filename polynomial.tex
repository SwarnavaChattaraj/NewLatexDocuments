\documentclass{scrartcl} % KOMA-Script class for automatic colors
\usepackage[sexy]{evan} % Load evan.sty with 'sexy' option
\usepackage{amsmath}    % For math formatting
\usepackage{xcolor}     % Ensure xcolor is available (often needed)

\title{Introduction to Polynomials}
\author{Swarnava Chattaraj}
\date{\today}

\begin{document}

\maketitle

\section{Introduction} % Colored section (automatically from 'sexy' + scrartcl)
\begin{theorem}[Binomial Theorem]
For any non-negative integer \( n \),
\[ 2^n = \sum_{i = 0}^{n} \binom{n}{i} \]
\end{theorem}
\newpage
\begin{example}
Let $f(x) = x^2 + 6x + c, \forall\ x \in \RR, c \in \RR$
Find the value of $c$ for which $f(f(x))$ has 3
distinct real roots.
\end{example}
\begin{proof}[Solution]
Notice that $deg\left(f(f(x))\right) = 4$. The condition
wants $f(f(x))$ to have only 3 distinct real root, so
the fourth root cannot be a complex root $($Recall that
complex roots occur in pairs$)$, hence it must be a
repeated root.\\
Suppose $f(f(x))$ the three real distinct roots be
$\alpha_1, \alpha_2, \alpha_3$, let the roots of $f(x)$
be $r_1, r_2$.
Then we can say that \[
	f : \{\alpha_1, \alpha_2, \alpha_3\}
\rightarrow \{r_1, r_2\}.
\]
Since $f$ is a quadratic polynomial, hence $f(x) = t$, 
can have at most solutions.\\
So if $r_1 = r_2$, and $f$ maps three numbers to a 
single number, then by \\ \textbf{Pigeon Hole Principle}
one of them is repeating.\\
The other case is $r_1$ and $r_2$ are distinct,
\begin{align*}
	f &: \{\alpha_{i_1}, \alpha_{i_2}\} \rightarrow
 \{r_{j_1}\}\\
	f &: \{\alpha_{i_3}\} \rightarrow \{r_{j_2}\}
\end{align*}
such that there exists some permutation maps
\begin{align*}
	\sigma_1&: \{i_1,i_2,i_3\} \rightarrow \{1,2,3\}
	\\
	\sigma_2&: \{j_1,j_2\} \rightarrow \{1,2\}
\end{align*}
WLOG, we can set $\sigma_1 \coloneq 1$ and
$\sigma_2 \coloneq 1$ or in other words we can set the permutation maps to be the identity maps.\\
So the following claims hold true:
\begin{align*}
	f(x) &= r_2 \text{ $($have double root$)$}\\
	f(x) &= r_1 \text{ $($have 2 distinct root$)$}
\end{align*}
Now if $f(x) = r_2$ has a double root then 
$g(x) \coloneq f(x) - r_2$ is a square polynomial.\\
Inferring from the structure of $(ax)^2 + 2(a)(b)x + b^2
$, we can say that \\$g(x) = x^2 + 6x + (c-r_1) =(x - 3)^2
$.\\
Now $(c - r_1) = 9 \implies r_1 = c_1 - 9$. Substitute $r_1 = c_1 - 9$ in the equation $f(r_1) = 0$ and find the
value of $c$. If there are multiple values of $c$, check
which works.\\
\textbf{Caution}, there are two ways find $f(x)$ for
both the value of $c$, compute $f(f(x))$ and see if they
have 3 distinct real roots or you can be clever about it
.
\end{proof}
\end{document}
